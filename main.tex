\documentclass[12pt,UTF8]{article}
\usepackage[twoside,a4paper,inner=0.5in,outer=1in,bindingoffset=0.5in,top=1.5in,bottom=1.5in]{geometry}
\usepackage{indentfirst}
\usepackage[onehalfspacing]{setspace}
\usepackage{xcolor}
\usepackage{ctex}
\usepackage{amsmath,amssymb}
\usepackage{ragged2e}
\usepackage{eso-pic}
\usepackage{graphicx}
\usepackage{ulem}
\usepackage{fancyhdr}
\usepackage{hyperref}
\usepackage{hyperxmp}
\usepackage{textcomp}

\hypersetup{
colorlinks=true,urlcolor=blue,%杂项结束,我要开始装逼了
%pdf元数据
pdftitle={2025-11-13思沁模考25(3)},%作品标题
pdfsubject={Mathematics},%作品主题
pdfauthor={gcc-404-NOT-FOUND},%作者是我
pdfkeywords={思沁,模拟考,期中},%搜索关键词
pdfcreator={gcc-404-NOT-FOUND},
pdfproducer={Overleaf® TeX Live 2025 XeLaTeX},%编译器使用TeX Live 2025
pdfcreationdate={\today},%修改日期:编译时的那天
pdfcopyright={Copyright (c) 2025 by gcc-404-NOT-FOUND.文章为gcc-404-NOT-FOUND原创,未经允许不可转卖或盗用。This article is originally written by gcc-404-NOT-FOUND. Unauthorized resale or misappropriation is prohibited.},%免责声明
pdfversion={v1.0.1}%版本号
}

%格式化
\renewcommand{\title}[1]{{\centering \huge{\textbf{\textsf{#1}}}\\}} %用\title{}打印主标题
\newcommand{\subtitle}[1]{{\RaggedLeft \large{\textsf{------#1}}\par}} %用\subtitle{}打印副标题
%页眉页脚
\fancyhf{}

\fancyhead[CO]{\Large{\LaTeX}}
\fancyhead[CE]{\includegraphics[scale=0.01]{Overleaf_Logo.jpeg} Overleaf\textsuperscript{\textregistered} \includegraphics[scale=0.01]{Overleaf_Logo.jpeg}}
\fancyfoot[LE,RO]{\thepage}
\fancyfoot[CO]{Get source code by emailing the author at TemplateCPP@Outlook.com}
\fancyfoot[CE]{\TeX\ Live 2025}

\pagestyle{fancy}
%水印
\AddToShipoutPictureBG{%水印
    \AtPageLowerLeft{%
        \rotatebox{54.737}{%
            \makebox[\paperwidth]{%
                \textcolor{gray!15}{\scalebox{1.25}{\Huge \texttt{Copyright \textsuperscript{\textcopyright} 2025 by gcc-404-NOT-FOUND}}} % 降低透明度,调整大小
            }%
        }%
    }%
}

\setlength{\parindent}{0em}%全局禁用首段缩进

\begin{document}
\title{解析几何法解25(3)}
\subtitle{湖南湘江新区思沁学校2025年下学期八年级数学期中模拟考试}

\section{原题}
25.如图1,点A在$y$轴正半轴上,点B在$x$轴负半轴上,点C和点D分别在第四象限和第一象限,OA=OB,OC=OD,OC$ \perp $OD,点D的坐标为$(m,n)$,且满足\(n^2 -4n+4+(2n-m)^2 =0\).\par
(3)如图2,点P,Q分别在$y$轴正半轴和$x$轴负半轴上,且OP=OQ,直线ON$ \perp $BP交AB于点N,MN$ \perp $AQ交BP的延长线于点M,判断ON,MN,BM的数量关系并证明.
\begin{figure}[ht]
    \centering
    \includegraphics[width=0.5\textwidth]{IMG_2.jpeg}
    \addtocounter{figure}{1}
    \caption{图像由GeoGebra\textsuperscript{\textregistered}绘制}
    \label{fig:placeholder}
\end{figure}

\section{使用公式}
\begin{equation*}
    L=\int^a_b \sqrt{1+[\frac{\mathrm{d}f(x)}{\mathrm{d}x}]^2}\mathrm{d}x\quad (x\in[a,b])
    \label{eq:FunctionLength}
\end{equation*}

\section{解:}
\paragraph{BM=ON+MN.}理由:\par
设OP=OQ=$a$,AP=BQ=$b$\par
则很容易得到:
\begin{align}
    \overleftrightarrow{\mathrm{AB}}:y&=x+a+b
    \label{func:AB}\\
    \overleftrightarrow{\mathrm{BM}}:f_1(x)&=\frac{a}{a+b}x+a
    \label{func:BM}\\
    \overleftrightarrow{\mathrm{AQ}}:f_2(x)&=\frac{a+b}{a}x+a+b
    \label{func:AQ}
\end{align}
$\because$ON$\perp$BM\par
$\therefore$ON的斜率为BM的负倒数\par
$\therefore$有:
\begin{align}
    \label{func:ON}
    \overleftrightarrow{\mathrm{ON}}:f_3(x)=-\frac{a+b}{a}x
\end{align}
令\hyperref[func:ON]{$f_3(x)$}$=$\hyperref[func:AB]{$y$},得AB与ON的交点N有:
\begin{align*}
    -\frac{a+b}{a}x=x+a+b
\end{align*}
解得\(x=-\frac{a^2+ab}{2a+b}\)\par
则有N\((-\frac{a^2+ab}{2a+b},\frac{a^2+2ab+b^2}{2a+b})\)\par
$\because$MN$\perp$AQ\par
$\therefore$有
\begin{align}
    \label{func:MN}
    \overleftrightarrow{\mathrm{MN}}:f_4(x)=-\frac{a}{a+b}x+b
\end{align}
令\hyperref[func:BM]{$f_1(x)$}$=$\hyperref[func:MN]{$f_4(x)$},得BM与MN的交点M有:
\begin{align*}
    \frac{a}{a+b}x+a=-\frac{a}{a+b}x+b
\end{align*}
解得\(x=\frac{b^2-a^2}{2a}\)\par
则有M\((\frac{b^2-a^2}{2a},\frac{a+b}{2})\)\par
$\therefore$有:
\begin{align*}
    L_{\mathrm{\overline{ON}}}&=\int^0_{-\frac{a^2+ab}{2a+b}} \sqrt{1+[\frac{\mathrm{d}(-\frac{a+b}{a}x)}{\mathrm{d}x}]^2}\mathrm{d}x\\
    &=\frac{a+b}{2a+b} \sqrt{2a^2 + 2ab + b^2}\\
    L_{\overline{\mathrm{MN}}}&=\int^{\frac{b^2-a^2}{2a}}_{-\frac{a^2+ab}{2a+b}} \sqrt{1+[\frac{\mathrm{d}(-\frac{a}{a+b}x+b)}{\mathrm{d}x}]^2}\mathrm{d}x\\
    &=\frac{b(a+b)}{2a(2a+b)} \sqrt{2a^2 + 2ab + b^2}\\
    L_{\overline{\mathrm{BM}}}&=\int^{\frac{b^2-a^2}{2a}}_{-a-b} \sqrt{1+[\frac{\mathrm{d}(\frac{a}{a+b}x+a)}{\mathrm{d}x}]^2}\mathrm{d}x\\
    &=\frac{a+b}{2a} \sqrt{2a^2 + 2ab + b^2}
\end{align*}
\(\because\frac{a+b}{2a+b}+\frac{b(a+b)}{2a(2a+b)}=\frac{a+b}{2a}\)\par
\(\therefore L_{\mathrm{\overline{ON}}}+L_{\overline{\mathrm{MN}}}=L_{\overline{\mathrm{BM}}}\)
$\therefore$ON+MN=BM

\begin{thebibliography}{3}
\bibitem{} 《解析几何》\_丘维声著\_北京大学出版社\_ISBN:978-7-301-28005-8
\bibitem{} 《微分几何》第五版\_梅向明、黄敬之著\_高等教育出版社\_ ISBN:978-7-040-50741-6
\end{thebibliography}

\end{document}
